\documentclass{article}

% Use Packages ============================
\usepackage{amsmath}
\usepackage{amsfonts}
\usepackage{amssymb}
\usepackage{ragged2e}
\usepackage{listings}
\usepackage{multicol}
\usepackage{url}
\usepackage{hyperref}

% Listing Setup =============================
\lstset{
    language=python,
    numbers=left,
    breaklines=true,
    commentstyle=\color{green!50!black},
    keywordstyle=\color{blue}\bfseries,
    numberstyle=\tiny\color{black!30},
}

% Title ===================================
\title{Day 1 Quantum Computing Basics}
\author{Prakhar Mittal}

\begin{document}
    
\maketitle
\tableofcontents

\newpage

\section{Basic Linear Algebra}
The world of quanntum computing demands a small fee of understanding of the basics of 
Vector Spaces and Linear Algebra. So here's a short intro/revision of them.

    \subsection{Vector Spaces}
    A \emph{vector space} $V$ over the field of complex numbers $\mathbb{C}$ is a non-empty set of elements called vectors
    . In $V$, it is defined the operations of vector addition and multiplication of a vector by a scalar in
    $\mathbb{C}$. The addition operation is associative and commutative. It also obeys properties 
    \begin{itemize}
        \item There is an element $0 \in V$, such that, for each $\mathbf{v} \in V, \mathbf{v}+0=0+\mathbf{v}=\mathbf{v}$
        (existence of a neutral element) 
    \end{itemize}


\newpage

\section{Gates}
Just like the classical circuits, the quantum circuits will also have some gates like
the AND OR NOT gates, which would be used for computations between qubits.
The gates can be represented as martices.


\begin{multicols}{2}
    \begin{align*}
        \mathbf{Pauli-X} &= \begin{bmatrix}
                                0 & 1 \\
                                1 & 0
                            \end{bmatrix}\\
    \mathbf{Pauli-Y} &= \begin{bmatrix}
                                0 & -i \\
                                i & 0
                            \end{bmatrix}\\
    \mathbf{Pauli-Z} &= \begin{bmatrix}
                                0 & -1 \\
                                1 & 0
                            \end{bmatrix}\\
    \mathbf{I} &= \begin{bmatrix}
                                1 & 0 \\
                                0 & 1
                            \end{bmatrix}\\
    \mathbf{cnot}
    \end{align*}

\end{multicols}






\end{document}